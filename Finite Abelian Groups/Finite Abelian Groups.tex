\documentclass[a4paper,12pt]{article}
\usepackage{fullpage}
\usepackage{amsmath}
\usepackage{amsthm}
\usepackage{amsfonts}
\usepackage{setspace}

\begin{document}

%\begin{flushright}
%Dillon Wong \\
%\end{flushright}

\begin{center}
\mbox{} \\
{ \large Project 1: Finite Abelian Groups}
\end{center}

\onehalfspace

\newtheorem*{lemma}{Lemma}

\begin{lemma}
Suppose that $m=\sum{r_i \alpha_i}$ for some $r_i \in \{0,1,\cdots,p^{k_i}-1\}$ with $gcd(r_i,p)=1$ for all $i$.  To ease notation, assume that r$_1 \neq 0$ and $k_1 = max\{k_i | r_i \neq 0 \}$.  Then $M \cong \langle m \rangle \times M_2 \times M_3 \times \cdots \times M_n$.
\end{lemma}
\begin{proof}
Note that it is sufficient to prove that $M_1 \cong \langle m \rangle$, or equivalently, since both are cyclic groups, $p^{k_1} = |M_1| = |m|$. \\
The identity in $G$ will be denoted as $0$, the same symbol used to denote the additive identity in $\mathbb{Z}$.  Note that the identity in $\langle m \rangle$ is $(0,\cdots,0,\cdots,0)$.
\[ p^{k_1} m = p^{k_1} (r_1 x_1, \cdots, r_i x_i, \cdots, r_n x_n) = (r_1 (p^{k_1} x_1), \cdots, p^{k_1} r_i x_i, \cdots, p^{k_1} r_n x_n) \]
$p^{k_1} x_1 = 0$ because $x_1$ is a generator for $M_1$, so $|x_1|=p^{k_1}$. \\
And for all other $1 < i \leq n$, since $k_1 = max\{k_i | r_i \neq 0 \}$, $k_1 = k_i + \beta_i$, where $\beta_i \geq 0$. Then, $p^{k_1} r_i x_i = r_i p^{\beta_i} (p^{k_i} x_i) = r_i p^{\beta_i} (0) = 0$.  Thus,
\[ p^{k_1} m = (0,\cdots,0,\cdots,0) \]
Now let $y \in \mathbb{Z}$ such that $0<y<p^{k_1}$.  Then,
\[ y m = y (r_1 x_1, \cdots, r_i x_i, \cdots, r_n x_n) = (y r_1 x_1, \cdots, y r_i x_i, \cdots, y r_n x_n) \]
Now, $y$ has at most $k_1-1$ factors of $p$ in its unique prime factorization.  Also, since $gcd(r_1,p)=1$, $r_1$ has no factors of $p$.  Then, $y r_1$ has at most $k_1-1$ factors of $p$.  That is, $p^{k_1}$ does not divide $y r_1$.  Thus, with the assumption that $r_1 \neq 0$, $y r_1 x_1 \neq 0$, so
\[ y m \neq (0,\cdots,0,\cdots,0) \]
Therefore, $|m| = p^{k_1}$, and thus, $M \cong \langle m \rangle \times M_2 \times M_3 \times \cdots \times M_n$.
\end{proof}

\end{document}