\documentclass[a4paper,12pt]{article}
\usepackage{fullpage}
\usepackage{amsmath}
\usepackage{graphicx}
\usepackage{subfig}
\usepackage{setspace}

\begin{document}

%Dillon Wong
%\\

Is the Maxwell-Boltzmann distribution applicable to any arbitrary potential in nonrelativistic classical mechanics?  If we believe in equal {\it a priori} probabilities in the micro-canonical ensemble, then for reasonably well-behaved potentials that don't depend on momentum (and with no magnetic fields because a vector potential messes up the derivation), I think we should believe in the Maxwell-Boltzmann distribution.
\\  \\
In 3D, the distribution of velocities is proportional to
\begin{equation}
v^2 e^{-\beta m v^2/2}
\end{equation}
In 2D, we lose a factor of $v$ because we integrate over a circle instead of a sphere\begin{equation}
v e^{-\beta m v^2/2}
\end{equation}
And in 1D,
\begin{equation}
e^{-\beta m v^2/2}
\end{equation}
Let prove that the 1D simple harmonic oscillator obeys the Maxwell-Boltzmann distribution.  The Hamiltonian is
\begin{equation}
H=\frac{p^2}{2m}+\frac{1}{2}m \omega^2 x^2
\end{equation}
For simplicity, let's work in units where $m=\omega=1$.  Then the energy is
\begin{equation}
E=\frac{1}{2} (v^2+x^2)
\end{equation}
At energy E, and setting t=0 to coincide with zero speed, the velocity is given by
\begin{equation}
v(t)=\sqrt{2E}\sin(t)
\end{equation}
The velocity distribution of that oscillator is given by $P(v)dv=|2 dt/T|$, where $T=2\pi$ is the period, but I will drop constants whenever I want.  Using $dv=\sqrt{2E}\cos(t)$,
\begin{equation}
P(v)dv \propto \frac{dv}{\sqrt{2E}\cos(t)}=\frac{dv}{\sqrt{2E-(\sqrt{2E}\sin(t))^2}}=\frac{dv}{\sqrt{2E-v^2}}
\end{equation}
if $v^2<2E$.  Otherwise, $P(v)dv=0$.\\ \\
We can now take a collection of oscillators connected to a heat bath.  The probability that an oscillator has energy $E$ is proportional to $e^{-\beta E}$.  We can then multiply $P(v)dv$ by $e^{-\beta E}$ and integrate over energies.  The density of states is constant.  The limits of integration are $v^2/2$ to infinity because oscillators with energy less than $v^2/2$ are never at velocity $v$.
\begin{equation}
\int_{v^2/2}^\infty \frac{e^{-\beta E}}{\sqrt{2E-v^2}}\,\mathrm{d}E = \sqrt{\frac{\pi}{2\beta}} e^{-\beta v^2/2}
\end{equation}
And we've recovered the 1D Maxwell-Boltzmann distribution!

\end{document}