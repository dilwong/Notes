\documentclass[a4paper,12pt]{article}
\usepackage{fullpage}
\usepackage{amsmath}
\usepackage{graphicx}
\usepackage{subfig}
\usepackage{setspace}

\setlength{\parindent}{0pt} 

\begin{document}

\textbf{Tunneling into Bloch States} \\

According to the simple 1D model of tunneling through a square potential barrier, a sample state decays into vacuum as $\psi(z)=\psi(0) e^{-\kappa z}$, where $\kappa=(2m(\phi-E_s)/\hbar^2)^{1/2}$ and $\phi$ is the sample work function.  However, a correction must be made for Bloch states in a crystal.  For simplicity, suppose a Bloch state decaying into the vacuum has the form
\begin{equation}
\psi(x,y,z)=e^{i(k_x x + k_y y)} e^{-\gamma z}
\end{equation}
Then, plugging into the Schr\"{o}dinger Equation
\begin{equation}
\left( \frac{\partial^2}{\partial x^2}+\frac{\partial^2}{\partial z^2}+\frac{\partial^2}{\partial z^2} \right)\psi=\kappa^2 \psi
\end{equation}
we get
\begin{equation}
\gamma = \sqrt{\kappa^2 +k_{||}^2}
\end{equation}
where $k_{||}^2=k_x^2+k_y^2$ is the component of the crystal momentum parallel to the surface.  Thus, states with a larger parallel component of momentum will decay faster than states with a smaller parallel component of momentum.  This is very important in interpreting $dI/dV$ spectroscopy for graphene.

\end{document}